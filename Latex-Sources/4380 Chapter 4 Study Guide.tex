 \documentclass{article}
\usepackage{hyperref}
\usepackage{amsmath}
\usepackage{amssymb}
\usepackage{graphicx}
\usepackage{enumitem}

\title{Study Guide for Chapter 4 Access Control}

\begin{document}

\maketitle


\section{Access Control Principles}
\subsection{Access Control Context}
Access control is a fundamental aspect of computer security, ensuring that only authorized entities can access specific resources. It involves:
\begin{itemize}
    \item \textbf{Authentication}: Verifying the credentials of a user or system entity.
    \item \textbf{Authorization}: Granting rights or permissions to access resources.
    \item \textbf{Audit}: Reviewing system records to ensure compliance and detect security breaches.
\end{itemize}

\subsection{Access Control Policies}
Access control policies dictate who or what can access specific resources and the type of access permitted. Policies are categorized into:
\begin{itemize}
    \item \textbf{Discretionary Access Control (DAC)}: Access is based on the identity of the requestor and access rules.
    \item \textbf{Mandatory Access Control (MAC)}: Access is based on security labels and clearances.
    \item \textbf{Role-Based Access Control (RBAC)}: Access is based on user roles within the system.
    \item \textbf{Attribute-Based Access Control (ABAC)}: Access is based on attributes of the user, resource, and environment.
\end{itemize}

\section{Subjects, Objects, and Access Rights}
\subsection{Subjects}
A \textbf{subject} is an entity capable of accessing objects, typically equated with a process. Subjects are categorized into:
\begin{itemize}
    \item \textbf{Owner}: Creator or administrator of a resource.
    \item \textbf{Group}: Named group of users with specific access rights.
    \item \textbf{World}: Users not included in owner or group categories.
\end{itemize}

\subsection{Objects}
An \textbf{object} is a resource to which access is controlled, such as files, directories, or devices.

\subsection{Access Rights}
Access rights describe how a subject may access an object, including:
\begin{itemize}
    \item \textbf{Read}: View information.
    \item \textbf{Write}: Modify or delete data.
    \item \textbf{Execute}: Run programs.
    \item \textbf{Delete}: Remove resources.
    \item \textbf{Create}: Generate new resources.
    \item \textbf{Search}: List or search directories.
\end{itemize}

\section{Discretionary Access Control (DAC)}
\subsection{Access Control Model}
DAC allows entities to grant access rights to others. It is implemented using:
\begin{itemize}
    \item \textbf{Access Control Lists (ACLs)}: Lists users and their permitted access rights for each object.
    \item \textbf{Capability Tickets}: Specify authorized objects and operations for a user.
\end{itemize}

\subsection{Protection Domains}
A \textbf{protection domain} is a set of objects with access rights. Domains can be static or dynamic, allowing processes to have different access rights at different times.

\section{Example: Unix File Access Control}
\subsection{Traditional UNIX File Access Control}
UNIX uses a hierarchical file system with permissions for owner, group, and others. Special permissions include:
\begin{itemize}
    \item \textbf{SetUID}: Temporarily grants the user the rights of the file owner.
    \item \textbf{SetGID}: Temporarily grants the user the rights of the file group.
    \item \textbf{Sticky Bit}: Restricts file deletion to the owner.
\end{itemize}

\subsection{Access Control Lists in UNIX}
Modern UNIX systems support extended ACLs, allowing more flexible access control by assigning permissions to named users and groups.

\section{Role-Based Access Control (RBAC)}
\subsection{RBAC Reference Models}
RBAC assigns access rights to roles rather than individual users. The NIST RBAC standard defines four models:
\begin{itemize}
    \item \textbf{RBAC$_0$}: Base model with users, roles, permissions, and sessions.
    \item \textbf{RBAC$_1$}: Adds role hierarchies.
    \item \textbf{RBAC$_2$}: Adds constraints.
    \item \textbf{RBAC$_3$}: Combines RBAC$_1$ and RBAC$_2$.
\end{itemize}

\subsection{Role Hierarchies}
Role hierarchies allow roles to inherit permissions from subordinate roles, reflecting organizational structures.

\subsection{Constraints}
Constraints restrict role assignments and permissions, such as mutually exclusive roles and cardinality limits.

\section{Attribute-Based Access Control (ABAC)}
\subsection{Attributes}
ABAC uses attributes of subjects, objects, and the environment to make access control decisions. Attributes include:
\begin{itemize}
    \item \textbf{Subject Attributes}: Characteristics of the user or process.
    \item \textbf{Object Attributes}: Characteristics of the resource.
    \item \textbf{Environment Attributes}: Contextual information like time or location.
\end{itemize}

\subsection{ABAC Logical Architecture}
ABAC evaluates access requests based on predefined rules and attributes, providing fine-grained access control.

\subsection{ABAC Policies}
Policies define rules for access based on attributes, allowing for flexible and dynamic access control.

\section{Identity, Credential, and Access Management (ICAM)}
\subsection{Identity Management}
ICAM manages digital identities and attributes, ensuring trustworthy identities across applications.

\subsection{Credential Management}
Credentials bind identities to tokens, such as smart cards or digital certificates, and are managed throughout their lifecycle.

\subsection{Access Management}
Access management ensures that entities are granted appropriate access to resources based on their identity and attributes.

\subsection{Identity Federation}
Identity federation allows organizations to trust digital identities and attributes from external sources, facilitating collaboration.

\section{Trust Frameworks}
\subsection{Traditional Identity Exchange Approach}
Traditional identity exchange involves agreements between identity service providers and relying parties, ensuring trust in shared identity information.

\subsection{Open Identity Trust Framework (OITF)}
OITF provides a standardized approach to identity and attribute exchange, ensuring trust and security in digital transactions.

\section{Case Study: RBAC System for a Bank}
The Dresdner Bank implemented an RBAC system to manage access to various applications. Roles were defined by job function and position, with access rights assigned based on roles. The system improved security and reduced administrative overhead.

\section{Recommended Reading}
Key references for further study include:
\begin{itemize}
    \item \textbf{[DOWN85]}: Basic elements of DAC.
    \item \textbf{[SAND96]}: Comprehensive overview of RBAC.
    \item \textbf{[HU13]}: Overview of ABAC models.
    \item \textbf{[CIOC11]}: Introduction to ICAM.
    \item \textbf{[RUND10]}: Overview of OITF.
\end{itemize}

\section{Key Terms}
Key terms include:
\begin{itemize}
    \item \textbf{Access Control List (ACL)}: A list of permissions attached to an object.
    \item \textbf{Role-Based Access Control (RBAC)}: Access control based on user roles.
    \item \textbf{Attribute-Based Access Control (ABAC)}: Access control based on attributes.
    \item \textbf{Identity Federation}: Trusting digital identities from external organizations.
    \item \textbf{Protection Domain}: A set of objects with associated access rights.
\end{itemize}

\newpage
\section*{Multiple Choice Questions}


\begin{enumerate}
    \item Which of the following is NOT a category of access control policies?
    \begin{enumerate}[label=\Alph*)]
        \item Discretionary Access Control (DAC)
        \item Mandatory Access Control (MAC)
        \item Role-Based Access Control (RBAC)
        \item User-Based Access Control (UBAC)
    \end{enumerate}
    \textbf{Answer: D}

    \item In the context of access control, what is a subject?
    \begin{enumerate}[label=\Alph*)]
        \item A resource to which access is controlled
        \item An entity capable of accessing objects
        \item A set of rules governing access
        \item A type of access right
    \end{enumerate}
    \textbf{Answer: B}

    \item Which of the following is an example of an access right?
    \begin{enumerate}[label=\Alph*)]
        \item Read
        \item Write
        \item Execute
        \item All of the above
    \end{enumerate}
    \textbf{Answer: D}

    \item What is the primary characteristic of Discretionary Access Control (DAC)?
    \begin{enumerate}[label=\Alph*)]
        \item Access is based on security labels
        \item Access is based on the identity of the requestor
        \item Access is based on user roles
        \item Access is based on environmental conditions
    \end{enumerate}
    \textbf{Answer: B}

    \item Which of the following is true about Role-Based Access Control (RBAC)?
    \begin{enumerate}[label=\Alph*)]
        \item Access rights are assigned to individual users
        \item Access rights are assigned to roles
        \item Access rights are based on security labels
        \item Access rights are based on environmental conditions
    \end{enumerate}
    \textbf{Answer: B}

    \item What is a protection domain in access control?
    \begin{enumerate}[label=\Alph*)]
        \item A set of objects with associated access rights
        \item A list of users and their permissions
        \item A type of access control policy
        \item A security label
    \end{enumerate}
    \textbf{Answer: A}

    \item In UNIX file access control, what does the SetUID permission do?
    \begin{enumerate}[label=\Alph*)]
        \item Grants the user the rights of the file owner
        \item Grants the user the rights of the file group
        \item Restricts file deletion to the owner
        \item Allows the file to be executed as a program
    \end{enumerate}
    \textbf{Answer: A}

    \item Which of the following is a key element of Attribute-Based Access Control (ABAC)?
    \begin{enumerate}[label=\Alph*)]
        \item Attributes
        \item Roles
        \item Security labels
        \item Access control lists
    \end{enumerate}
    \textbf{Answer: A}

    \item What is the purpose of identity federation in access control?
    \begin{enumerate}[label=\Alph*)]
        \item To create digital identities for users
        \item To manage credentials throughout their lifecycle
        \item To trust digital identities from external organizations
        \item To define access control policies
    \end{enumerate}
    \textbf{Answer: C}

    \item Which of the following is a type of constraint in RBAC?
    \begin{enumerate}[label=\Alph*)]
        \item Mutually exclusive roles
        \item Role hierarchies
        \item Access control lists
        \item Protection domains
    \end{enumerate}
    \textbf{Answer: A}

    \item What is the primary function of an access control list (ACL)?
    \begin{enumerate}[label=\Alph*)]
        \item To define roles and their permissions
        \item To list users and their permitted access rights for an object
        \item To manage credentials
        \item To define protection domains
    \end{enumerate}
    \textbf{Answer: B}

    \item Which of the following is true about the NIST RBAC standard?
    \begin{enumerate}[label=\Alph*)]
        \item It defines four models: RBAC$_0$, RBAC$_1$, RBAC$_2$, and RBAC$_3$
        \item It is based on security labels
        \item It is primarily used in UNIX systems
        \item It does not support role hierarchies
    \end{enumerate}
    \textbf{Answer: A}

    \item What is the main advantage of Attribute-Based Access Control (ABAC) over Role-Based Access Control (RBAC)?
    \begin{enumerate}[label=\Alph*)]
        \item ABAC is simpler to implement
        \item ABAC provides finer-grained access control
        \item ABAC does not require attributes
        \item ABAC is based on security labels
    \end{enumerate}
    \textbf{Answer: B}

    \item In the context of access control, what is a credential?
    \begin{enumerate}[label=\Alph*)]
        \item A set of rules governing access
        \item An object that binds an identity to a token
        \item A type of access right
        \item A security label
    \end{enumerate}
    \textbf{Answer: B}

    \item Which of the following is a key component of Identity, Credential, and Access Management (ICAM)?
    \begin{enumerate}[label=\Alph*)]
        \item Access control lists
        \item Role hierarchies
        \item Credential management
        \item Protection domains
    \end{enumerate}
    \textbf{Answer: C}
\end{enumerate}

\

\end{document}